\documentclass[handouts,hyperref={pdfpagelabels=false}]{beamer}
\let\Tiny=\tiny
\usetheme{Copenhagen}
\usepackage[T1]{fontenc}
\usepackage[utf8]{inputenc}
\usepackage[greek,english]{babel}
\usepackage[fixlanguage]{babelbib}
\usepackage{hyperref}
\selectbiblanguage{greek}
\graphicspath{ {images/} }

\hypersetup{
  unicode=true,
  colorlinks=true,
  % linkcolor=green,
  citecolor=red,
  % filecolor=blue,
  urlcolor=blue,
  % pdftitle=,
  % pdfauthor=,
  % pdfsubject=,
  % pdfkeywords=
}

\begin{document}
\selectlanguage{greek}

\title{Διαχείριση Τεχνολογιών στο Ηλεκτρονικό Εμπόριο}
\author{Δημήτριος Πολίτης}
\institute{
  Εθνικό Μετσόβιο Πολυτεχνείο\\
  Σχολή Ηλεκτρολόγων Μηχανικών και Μηχανικών Η/Υ\\
}

\begin{frame}
\titlepage
\end{frame}

\begin{frame}
\frametitle{Περίγραμμα Παρουσίασης}
\tableofcontents
\end{frame}

\section{\foreignlanguage{greek}{Εισαγωγή}}
\begin{frame}
\frametitle{\foreignlanguage{greek}{Ηλεκτρονικό Εμπόριο}}

\begin{block}{Ηλεκτρονικό Εμπόριο}
\begin{itemize}
  \item Είναι η αγοραπωλησία αγαθών και υπηρεσιών μέσω ενός ηλεκτρονικού μέσου, όπως το Διαδίκτυο.
  \item Είναι γνωστό και ως Ηλεκτρονικό Μάρκετινγκ. Πολλές φορές συγχέεται με το Ηλεκτρονικό Επιχειρείν, το οποίο όμως αποτελεί ευρύτερη έννοια, η οποία αναφέρεται επίσης στη εξυπηρέτηση πελατών, συνεργασία με επιχειρηματικούς εταίρους κ.α.~\cite{turban_outland_king_lee_liang_turban_2018}.
\end{block}

\begin{block}{Τύποι Ηλεκτρονικού Εμπορίου}
\begin{itemize}
  \item \textlatin{Bussiness to Bussiness  (B2B)}
  \item \textlatin{Bussiness to Consumer   (B2C)}
  \item \textlatin{Bussiness to Government (B2G)}
  \item \textlatin{Government to Bussiness (G2B)}
  \item \textlatin{Consumer to Consumer    (C2C)}
  \item \textlatin{Government to Consumer  (G2C)}
\end{itemize}
\end{block}
\end{frame}

\section{\foreignlanguage{greek}{Ηλεκτρονικές Θέσεις Αγορών}}
\begin{frame}
\frametitle{\foreignlanguage{greek}{Ηλεκτρονικές Αγορές}}
\begin{block}{Διάκριση Ηλεκτρονικών θέσεων αγορών (\textlatin{e-marketplace})}
Η βασική θέση για τη διεξαγωγή συναλλαγών ΗΕ είναι η ηλεκτρονική αγορά. Μια Ηλεκτρονική θέση Αγορών είναι μια εικονική θέση αγορών, στην οποία συναντώνται και διεξάγουν διάφορους τύπους συναλλαγών πωλητές και αγοραστές. Διακρίνονται σε:
\begin{itemize}
  \item Ιδιωτικές. Η θέση αγορών ανήκει σε ένα μόνο φορέα - εταιρία, η οποία τη διαχειρίζεται και είτε πωλεί τα δικά της προϊόντα (ένας προς πολλούς), είτε προσκαλεί προμηθευτές (πολλοί προς έναν).
  \item Δημόσιες. Ανήκουν σε κάποιο τρίτο και περιλαμβάνουν πολλούς αγοραστές και πωλητές.
  \item Θέσεις αγορών με πράκτορες. Συγκεντρώνουν πληροφορίες σχετικά με προϊόντα και τιμές.
\end{itemize}
Ένας από τους πιο ενδιαφέροντες μηχανισμούς της αγοράς στο ηλεκτρονικό εμπόριο είναι οι ηλεκτρονικές δημοπρασίες. Αυτές Χρησιμοποιούνται στα \textlatin{B2C, B2B, C2C, G2B, G2C} κ.α.
\end{block}
\end{frame}

\section{\foreignlanguage{greek}{Οι Δημοπρασίες ως Μηχανισμοί Αγορών ΗΕ}}
\begin{frame}
\frametitle{\foreignlanguage{greek}{Ορισμός}}
\begin{block}{Ορισμοί και χαρακτηριστικά}
Μια δημοπρασία είναι ένας μηχανισμός αγοράς, που χρησιμοποιεί μια ανταγωνιστική διαδικασία κατά την οποία ένας πωλητής δέχεται ακολουθιακές προσφορές από αγοραστές (προωθητική δημοπρασία) ή ένας αγοραστής δέχεται προσφορές από πωλητές (αντίστροφη δημοπρασία)~\cite{turban_outland_king_lee_liang_turban_2018}. Μπορούν να γίνουν:
\begin{itemize}
  \item Σε δημόσιες τοποθεσίες δημοπρασιών, όπως το \textlatin{EBay}.
  \item Κατόπιν προσκλήσης σε ιδιωτικές δημοπρασίες.
\end{itemize}
Στα επόμενα θα παρουσιαστεί η κατασκευή και η λειτουργικότητα μιας ιστοσελίδας δημοπρασιών.
\end{block}
\end{frame}

\begin{frame}
\frametitle{\foreignlanguage{greek}{Απειλές Ασύρματων Δικτύων}}
\begin{itemize}
    \item \textlatin{Silencing}
    \item \textlatin{Spoofing}
    \item \textlatin{Sybil attack}
    \item \textlatin{Jamming}
    \item \textlatin{Tampering}
    \item \textlatin{Node capture}
    \item \textlatin{Sinkhole attack}
    \item \textlatin{Denial of service}
    \item \textlatin{Selective forwarding}
    \item \textlatin{Wormhole attack}
    \item \textlatin{Blackhole attack}
    \item \textlatin{Routing request flooding attack}
    \item \textlatin{Routing request disrupt attack}
    \item \textlatin{Eavesdropping}
\end{itemize}
\end{frame}

\section{\foreignlanguage{greek}{Ανίχνευση Εισβολών με Ανάλυση της κίνησης}}
\begin{frame}
\frametitle{\foreignlanguage{greek}{Συστήματα \textlatin{IDS}}}
\begin{block}{Κύριες Λειτουργίες \textlatin{IDS}}
\begin{itemize}
    \item Συλλογή δεδομένων δικτυακής κίνησης
    \item Ανάλυση δεδομένων με χρήση τεχνικών ανίχνευσης επιθέσεων
\end{itemize}
\end{block}

\begin{block}{Μετρικές Αποτελεσματικότητας \textlatin{IDS}}
\begin{itemize}
    \item \textlatin{False Positive Rate (FPR)}
    \item \textlatin{False Negative Rate (FNR)}
    \item \textlatin{Detection Rate (DR)}
\end{itemize}
Η ευαισθησία του συστήματος ανίχνευσης υπολογίζεται από τη σχέση:
\[DR = 1 - FPR - FNR\]
\end{block}
\end{frame}

\begin{frame}
\frametitle{\foreignlanguage{greek}{Συλλογή και Ανάλυση Δεδομένων}}
\begin{block}{Συλλογή Δεδομένων}
\begin{itemize}
    \item \textlatin{traffic based collection}
    \item \textlatin{behavior based collection}
\end{itemize}
\end{block}
\begin{figure}
\includegraphics[scale=0.6]{coll-appr}
\caption{Διαδικασία συλλογής δεδομένων}
\end{figure}
\end{frame}

\begin{frame}
\frametitle{\foreignlanguage{greek}{Συλλογή και Ανάλυση Δεδομένων}}
\begin{block}{Ανάλυση Δεδομένων}
\begin{itemize}
    \item \textlatin{data minning}
    \item \textlatin{pattern matching}
\end{itemize}
\end{block}
\begin{figure}
\includegraphics[scale=0.6]{anal-appr}
\caption{Διαδικασία ανάλυσης δεδομένων}
\end{figure}
\end{frame}

\begin{frame}
\frametitle{\foreignlanguage{greek}{Ανίχνευση Επιθέσεων}}
\begin{block}{Ανίχνευση Επιθέσεων}
\begin{itemize}
    \item \textlatin{anomaly based}
    \item \textlatin{specification based}
    \item \textlatin{reputation based management}
    \item \textlatin{signature based}
\end{itemize}
\end{block}
\end{frame}

\begin{frame}
\frametitle{\foreignlanguage{greek}{Τύποι Ανωμαλιών στη Δικτυακή Κίνηση}}
\begin{block}{Κύριοι Τύποι αποκλίσεων}
\begin{itemize}
    \item \textlatin{point anomalies}
    \item \textlatin{context anomalies}
    \item \textlatin{collective anomalies}
\end{itemize}
\end{block}
\end{frame}

\begin{frame}
\frametitle{\foreignlanguage{greek}{Τρόποι Λειτουργίας Τεχνικών Ανίχνευσης}}
\begin{block}{Με βάση τα δεδομένα εκπαίδευσης}
\begin{itemize}
    \item \textlatin{Supervised Methods}
    \item \textlatin{Semisupervised Methods}
    \item \textlatin{Unsupervised Methods}
\end{itemize}
\end{block}

\begin{block}{Με βάση τον αλγόριθμο ανίχνευσης}
\begin{itemize}
    \item \textlatin{Classification based (Neural Networks, Bayesian Networks, Support Vector Machines, Rule-Based)}
    \item \textlatin{Nearest Neighboor Based}
    \item \textlatin{Clustering Based}
    \item \textlatin{Statistical Based (Parametric, non-Parametric)}
    \item \textlatin{Information Theory Based}
\end{itemize}
\end{block}
\end{frame}

\begin{frame}
\frametitle{\foreignlanguage{greek}{Αποτελεσματικότητα Ανίχνευσης \textlatin{anomaly based}}}
\begin{block}{Πλεονεκτήματα}
\begin{itemize}
    \item Δεν αναζητούν κάτι συγκεκριμένο - οχι \textlatin{attack dictionaries}
    \item Μεγάλο πλήθος εφαρμογών
    \item Ενδείκνειται για ασύρματους κόμβους με μικρό αποθηκευτικό χώρο
\end{itemize}
\end{block}

\begin{block}{Μειονεκτήματα}
\begin{itemize}
    \item Yψηλό \textlatin{FPR}
    \item Δυσκολία κατά τη δημιουργία του μοντέλου και των δεδομένων εκπαίδευσης
\end{itemize}
\end{block}
\end{frame}

\begin{frame}[allowframebreaks]
\frametitle{\foreignlanguage{greek}{Βιβλιογραφία}}
\nocite{*}
\bibliographystyle{babplain}
\bibliography{e-commerce}
\end{frame}

\end{document}